% Options for packages loaded elsewhere
\PassOptionsToPackage{unicode}{hyperref}
\PassOptionsToPackage{hyphens}{url}
%
\documentclass[
]{article}
\title{Mathematical notation - help my lecturer is speaking a different
language!}
\author{}
\date{\vspace{-2.5em}}

\usepackage{amsmath,amssymb}
\usepackage{lmodern}
\usepackage{iftex}
\ifPDFTeX
  \usepackage[T1]{fontenc}
  \usepackage[utf8]{inputenc}
  \usepackage{textcomp} % provide euro and other symbols
\else % if luatex or xetex
  \usepackage{unicode-math}
  \defaultfontfeatures{Scale=MatchLowercase}
  \defaultfontfeatures[\rmfamily]{Ligatures=TeX,Scale=1}
\fi
% Use upquote if available, for straight quotes in verbatim environments
\IfFileExists{upquote.sty}{\usepackage{upquote}}{}
\IfFileExists{microtype.sty}{% use microtype if available
  \usepackage[]{microtype}
  \UseMicrotypeSet[protrusion]{basicmath} % disable protrusion for tt fonts
}{}
\makeatletter
\@ifundefined{KOMAClassName}{% if non-KOMA class
  \IfFileExists{parskip.sty}{%
    \usepackage{parskip}
  }{% else
    \setlength{\parindent}{0pt}
    \setlength{\parskip}{6pt plus 2pt minus 1pt}}
}{% if KOMA class
  \KOMAoptions{parskip=half}}
\makeatother
\usepackage{xcolor}
\IfFileExists{xurl.sty}{\usepackage{xurl}}{} % add URL line breaks if available
\IfFileExists{bookmark.sty}{\usepackage{bookmark}}{\usepackage{hyperref}}
\hypersetup{
  pdftitle={Mathematical notation - help my lecturer is speaking a different language!},
  hidelinks,
  pdfcreator={LaTeX via pandoc}}
\urlstyle{same} % disable monospaced font for URLs
\usepackage[margin=1in]{geometry}
\usepackage{graphicx}
\makeatletter
\def\maxwidth{\ifdim\Gin@nat@width>\linewidth\linewidth\else\Gin@nat@width\fi}
\def\maxheight{\ifdim\Gin@nat@height>\textheight\textheight\else\Gin@nat@height\fi}
\makeatother
% Scale images if necessary, so that they will not overflow the page
% margins by default, and it is still possible to overwrite the defaults
% using explicit options in \includegraphics[width, height, ...]{}
\setkeys{Gin}{width=\maxwidth,height=\maxheight,keepaspectratio}
% Set default figure placement to htbp
\makeatletter
\def\fps@figure{htbp}
\makeatother
\setlength{\emergencystretch}{3em} % prevent overfull lines
\providecommand{\tightlist}{%
  \setlength{\itemsep}{0pt}\setlength{\parskip}{0pt}}
\setcounter{secnumdepth}{-\maxdimen} % remove section numbering
\usepackage{bbm}
\usepackage{undertilde}
\usepackage{accent}
\usepackage{tikz}
\usepackage{pgfplots}
\usepackage{DiagrammeR}
\ifLuaTeX
  \usepackage{selnolig}  % disable illegal ligatures
\fi

\begin{document}
\maketitle

There are lots of different ways of writing maths that are equivalent
and can be used interchangeably. This can be confusing if your lecturer
uses a different notation than you are used to.

The reason that there are different ways to say the same thing in maths
is because mathematical notation has been developed over many thousands
of years by many different people.

In English `big' (from Middle English) and `large' (from Latin) mean the
same thing. There are similar examples in maths.

The most important thing to remember is that choice of notation is often
cultural or simply personal preference, and therefore it is okay to ask
for clarification.

\hypertarget{numbers-and-calculations}{%
\subsection{Numbers and calculations}\label{numbers-and-calculations}}

\hypertarget{division-and-multiplication}{%
\subsubsection{Division and
multiplication}\label{division-and-multiplication}}

Sometimes we think of \(\frac{a}{b}\) as a `fraction' and \(a\div b\) as
a `calculation', but they are actually interchangeable.

\begin{equation}

\frac{2}{5} = 2\div 5 = 2/5  

\end{equation}

When working with numbers we often use a multiplication sign \(\times\),
but this is usually dropped when working algebraically as it looks a bit
like the letter x.

If you are working with vectors then then it is important to distinguish
between the cross product \(a \times b\) and the dot product
\(a \cdot b\), but with scalars (e.g.~straightforward numbers), they are
interchangeable.

\begin{equation}
a \times b = a \cdot b = ab = a(b) = (a)b = (a)(b) 

\end{equation}

\hypertarget{dots-and-commas}{%
\subsubsection{Dots and commas}\label{dots-and-commas}}

Different cultures tend to use dots and commas in different ways, if you
aren't sure please just ask your lecturer to clarify their notation.

A decimal point may be shown in several ways.

\begin{array}{ll}

2 \frac{3}{10} & = 2 \cdot 3 \\
& = 2.3 \\
& = 2,3

\end{array}

but a dot can also be used as multiplication

\begin{array}{ll}
2 \times 3 & = 2 \cdot 3 \\
& = 2.3 \\
& = 2(3)
\end{array}

Commas and dots might also be used for separating groups of a thousand

\begin{array}{ll}

2145600 & = 2 145 600 \\
& = 2,145,600 \\
& = 2.145.600 
\end{array}

\hypertarget{calculus}{%
\subsection{Calculus}\label{calculus}}

\hypertarget{equivalent-ways-of-showing-differentiation}{%
\subsubsection{Equivalent ways of showing
differentiation}\label{equivalent-ways-of-showing-differentiation}}

\begin{array}{c|c|c}
\textbf{function} & f(x)= x^2    & y= x^2 & y = x^2 & u^2 & s = t^2 & v = z^2 & f(x) = x^2 \\
 & \color{orange}{\downarrow} & \color{blue}{\downarrow} & \color{green}{\downarrow} & \color{red}{\downarrow} & \color{purple}{\downarrow} & \color{olive}{\downarrow} & \color{pink}{\downarrow} \\
\textbf{differential} & f'(x)= 2x   & \frac{dy}{dx}=2x &   y' = 2x & \frac{d(u^2)}{du}=2u & \frac{d}{dt}s = 2t & \dot{v} = 2z  & D(f) = 2x \\
 & \color{orange}{\downarrow} & \color{blue}{\downarrow} & \color{green}{\downarrow} & \color{red}{\downarrow} & \color{purple}{\downarrow} & \color{olive}{\downarrow} & \color{pink}{\downarrow}\\
\textbf{second differential} & f''(x) = 2 & \frac{d^2y}{dx^2}=2 & y'' = 2 & \frac{d^2(u^2)}{du^2}=2 & \frac{d^2}{dt^2}s = 2 & \ddot{v} = 2 & D^2(f) = 2

\end{array}

\hypertarget{letters-that-look-like-d-but-mean-different-things}{%
\subsubsection{Letters that look like d but mean different
things}\label{letters-that-look-like-d-but-mean-different-things}}

\begin{array}{lll}
\Delta{x} & \text{"delta x"} & \text{the change in x} \\
\delta{x} & \text{"delta x"} & \text{the small change in x} \\
d{x} & \text{"the differential"} & \text{the change in x when it approaches zero} \\
\partial{x} &  \text{"the partial differential"} & \text{the change in x when it approcahes zero} \\
d(x,y)  & \text{"distance between x and y"} & \text{the distance function (metric spaces and topology)}

\end{array}

\hypertarget{exponentials}{%
\subsection{Exponentials}\label{exponentials}}

In the expression \(a^b\), \(b\) can be called the `power' the
`exponent' or the `index' (indices is the plural of index).

In the expression \(e^b\), \(e\) is a number that might be called the
`natural number' or `Euler's number' (which is pronounced `oy-luh').

The expression \(e^x\) is a function, and can also be written in
function notation:

\begin{equation}
e^x = \text{exp}(x) = \text{exp}x
\end{equation}

The choice of notation might be used to emphasise that it is a function,
or perhaps just to make it easier to read:

\begin{equation}
e^{\frac{2x^2}{3 - x}} = \text{exp}\big({\frac{2x^2}{3 - x}}\big)
\end{equation}

\hypertarget{vectors}{%
\subsection{Vectors}\label{vectors}}

There are so many different notations that can be used for vectors, it
can be very confusing.

Vectors are often written in lowercase, and vertices are often
uppercase.

You can use a single letter to show a vector, or split it into its
components.

\begin{array}{lllll}
\vec{AB} & = \textbf{a} & = \underset{^\sim}{a} & = \vec{a} & = \hat{a} & = \underline{a}\\
= (a_1, a_2) & = (a_1 \space a_2) & =  \begin{pmatrix} a_1 \\ a_2 \end{pmatrix} & = \begin{bmatrix} a_1 \\ a_2 \end{bmatrix} &\\
= a_1 \textbf{i} + a_2 \textbf{j} & = a_1 \hat{i} + a_2 \hat{j} &
\end{array}

Watch out for pointy brackets though, as they are not used to show
something is a vector, but for an operation involving two vectors (the
inner product)

\begin{array}{lll}
\mathbf{a} = (a_1,a_2) & \mathbf{b} = (b_1,b_2) & \text{are vectors}\\
\langle \mathbf{a}, \mathbf{b} \rangle && \text{is the inner product of the two vectors}
\end{array}

\hypertarget{matrices}{%
\subsection{Matrices}\label{matrices}}

Matrices are usually named with a letter, which is in written in bold
font. You can use curved or square brackets for a matrix, or sometimes
even no bracket at all.

Another word for (curved) brackets is `parentheses' and \{curly\}
brackets are sometimes called `braces'.

\begin{equation}
\textbf{A} = 
\begin{pmatrix}
1 & 2 & 3 \\
4 & 5 & 6 \\
7 & 8 & 9
\end{pmatrix}
= 
\begin{bmatrix}
1 & 2 & 3 \\
4 & 5 & 6 \\
7 & 8 & 9
\end{bmatrix}
=
\begin{matrix}
1 & 2 & 3 \\
4 & 5 & 6 \\
7 & 8 & 9
\end{matrix}
\end{equation}

As a general rule matrices are written with uppercase letters,\(A\), and
elements of a matrix are denoted with lower case \(a\). Each element of
the matrix can be written with a subscript, \(a_{ij}\), that tells you
where it is located:

\begin{equation}
\textbf{A} = 
\begin{pmatrix}
a_{1,1} & a_{1,2} \\
a_{2,1} & a_{2,2}
\end{pmatrix}
\end{equation}

Curly brackets are not generally used for matrices, but because they are
used for set notation you could write a matrix using curly brackets:

\begin{equation}
\textbf{A} = \{a_{ij}\}
\end{equation}

\hypertarget{straight-lines}{%
\subsubsection{Straight lines}\label{straight-lines}}

Straight vertical lines are used for the property of the matrix, rather
than the matrix itself. Watch out - some lecturers' handwriting makes it
tricky to tell what style of line is intended! If in doubt, just ask for
clarification.

Single lines show the determinant of a matrix (telling you something
about the scale factor when using the matrix to enlarge a vector).

\begin{equation}
det(\textbf{A}) = 
\begin{vmatrix}
1 & 2 & 3 \\
4 & 5 & 6 \\
7 & 8 & 9
\end{vmatrix}
\ne
\begin{pmatrix}
1 & 2 & 3 \\
4 & 5 & 6 \\
7 & 8 & 9
\end{pmatrix}
\end{equation}

Double lines are for the norm of a matrix (telling you something about
the size of the elements).

\begin{equation}
norm(\textbf{A}) = 
\begin{Vmatrix}
1 & 2 & 3 \\
4 & 5 & 6 \\
7 & 8 & 9
\end{Vmatrix}
\ne
\begin{pmatrix}
1 & 2 & 3 \\
4 & 5 & 6 \\
7 & 8 & 9
\end{pmatrix}
\end{equation}

If you are working with numbers or vectors then single and double lines
are often used interchangeably to mean `distance from the origin'.

\begin{equation}
\Vert -3 \Vert = \vert -3 \vert = abs(-3) = 3 \\
\Vert (3,4) \Vert = \vert (3,4) \vert =  \sqrt{3^2 + 4^2} = 5 \\
\end{equation}

\hypertarget{statistics}{%
\subsection{Statistics}\label{statistics}}

\hypertarget{random-variables}{%
\subsubsection{Random variables}\label{random-variables}}

In general uppercase letters are used to name random variables and
lowercase letters are for the values those variables.

For example

\begin{array}{ll}

X & \text{possible values obtained when rolling a dice} \\
x & \text{the value after a single roll } \\
X = \{x\} &  \text{the set of possibly values} \\
& = \{1,2,3,4,5,6\} \\
x_1, x_2, x_3 & \text{the results obtained from three rolls of the dice} \\
x_i & \text{the value of one of the rolls of the dice} \\
P(X = x) & \text{the probability that the value rolled is a specific value} \\
P(X = 2) & \text{the probability that a 2 is rolled}

\end{array}

\hypertarget{summary-statistics}{%
\subsubsection{Summary statistics}\label{summary-statistics}}

\begin{array}{l\l}

\mu & \text{'mew'} &\text{population mean}\\
\mu_{X} & \text{'mew X'} &\text{population mean of X}\\
\bar{x} & \text{'x bar'} & \text{sample mean} \\
\sigma & \text{'sigma'} &\text{standard deviation}\\
\sigma_{X} & \text{'sigma X'} &\text{standard deviation of X}\\
s  && \text{sample standard deviation} \\
\rho & \text{'ro'} & \text{product moment correlation coefficient of population} \\
r_x && \text{product moment correlation coefficient of sample}




\end{array}

\hypertarget{dictionary}{%
\subsection{Dictionary}\label{dictionary}}

\hypertarget{general-notation}{%
\subsubsection{General notation}\label{general-notation}}

\begin{array}{ll}

= & \text{equal to} \\
\neq & \text{not equal to} \\
\approx & \text{is approximately equal to} \\
\infty & \text{infinity} \\ 
\propto & \text{is proportional to} \\
\therefore & \text{therefore} \\
\because & \text{because} \\
< & \text{less than} \\
> & \text{greater than} \\
\le & \text{less than or equal to} \\
\ge & \text{greater than or equal to} \\
\implies & \text{implies} \\ 
{:} & \text{such that} \\
\forall & \text{for all}

\end{array}

\hypertarget{set-notation}{%
\subsubsection{Set notation}\label{set-notation}}

\begin{array}{ll}

\in & \text{is an element of} \\
\notin & \text{is not an element of} \\
\subset & \text{is a subset of} \\
\emptyset & \text{the empty set} \\
A' & \text{the complement of} A \\
\mathbb{N} & \text{the natural numbers} \\
\mathbb{Z} & \text{the integers} \\
\mathbb{R} & \text{the real numbers} \\
\mathbb{Q} & \text{the rational numbers} \\
\mathbb{C} & \text{the complex numbers} \\
\cup & \text{union} \\
\cap & \text{intersection} \\
\end{array}

\hypertarget{series-notation}{%
\subsubsection{Series notation}\label{series-notation}}

\begin{array}{ll}

\sum & \text{sum of} \\
\sum_{i=1}^{n} a_n & = a_1 + a_2 + ... + a_n\\
\prod_{i=1}^{n} a_n & a_1 \times a_2 \times ... \times a_n \\
n! & \text{n factorial} \\
 & =n \times (n-1) \times... x 2 x 1 \\
\binom{n}{k} & \text{n choose r (the binomial coefficient)} \\
{}^{n}C_{k} & \text{n choose r (the binomial coefficient)}
\end{array}

\hypertarget{function-notation}{%
\subsubsection{Function notation}\label{function-notation}}

\begin{array}{ll}

f: x \mapsto y & \text{the function f maps x onto y} \\
f^{-1} & \text{the inverse of f} \\
\underset{x \rightarrow \infty}{lim} f(x) & \text{the limit of f(x) as x tends to infinity}

\end{array}

\end{document}
